% -*- compile-command: "cd ../ && make" -*-
\eocesolch{Inference for numerical data}

%%%%%%%%%%%%%%%%%%%%%%%%%%%%%

\begin{multicols}{2}

% 1
\eocesol{(a)~$df=6-1=5$, $t_{5}^{\star} = 2.02$ (column with two tails of 0.10, row with $df=5$).
(b)~$df=21-1=20$, $t_{20}^{\star} = 2.53$ (column with two tails of 0.02, row with $df=20$).
(c)~$df=28$, $t_{28}^{\star} = 2.05$.
(d)~$df=11$, $t_{11}^{\star} = 3.11$.}

% 3
\eocesol{(a)~between 0.025 and 0.05
(b)~less than 0.005
(c)~greater than 0.2
(d)~between 0.01 and 0.025}

% 5
\eocesol{The mean is the midpoint: $\bar{x} = 20$. Identify the margin of error: $ME = 1.015$, then use $t^{\star}_{35} = 2.03$ and $SE=s/\sqrt{n}$ in the formula for margin of error to identify $s = 3$.}

% 7
\eocesol{(a)~$H_0$: $\mu = 8$ (New Yorkers sleep 8 hrs per night on average.) $H_A$: $\mu < 8$ (New Yorkers sleep less than 8 hrs per night on average.)
(b)~Independence: The sample is random and from less than 10\% of New Yorkers. The sample is small, so we will use a $t$-distribution. For this size sample, slight skew is acceptable, and the min/max suggest there is not much skew in the data. $T = -1.75$. $df=25-1=24$.
(c)~$0.025 <$ p-value $<0.05$. If in fact the true population mean of the amount New Yorkers sleep per night was 8 hours, the probability of getting a random sample of 25 New Yorkers where the average amount of sleep is 7.73 hrs per night or less is between 0.025 and 0.05.
(d)~Since p-value $<$ 0.05, reject $H_0$. The data provide strong evidence that New Yorkers sleep less than 8 hours per night on average.
(e)~No, as we rejected $H_0$.}

% 9
\eocesol{$t^{\star}_{19}$ is 1.73 for a one-tail. We want the lower tail, so set -1.73 equal to the T-score, then solve for $\bar{x}$: 56.91.}


\textC{
\end{multicols}
\newpage
\begin{multicols}{2}
}


% 11
\eocesol{(a)~We will conduct a 1-sample $t$-test. $H_0$: $\mu = 5$. $H_A$: $\mu < 5$. We'll use $\alpha = 0.05$. This is a random sample, so the observations are independent. To proceed, we assume the distribution of years of piano lessons is approximately normal. $SE = 2.2 / \sqrt{20} = 0.4919$. The test statistic is $T = (4.6 - 5) / SE = -0.81$. $df = 20 - 1 = 19$. The one-tail p-value is about 0.21, which is bigger than $\alpha = 0.05$, so we do not reject $H_0$. That is, we do not have sufficiently strong evidence to reject Georgianna's claim. \\
(b)~Using $SE = 0.4919$ and $t_{df = 19}^{\star} = 2.093$, the confidence interval is (3.57, 5.63). We are 95\% confident that the average number of years a child takes piano lessons in this city is 3.57 to 5.63 years. \\
(c)~They agree, since we did not reject the null hypothesis and the null value of 5 was in the $t$-interval.}

% 13
\eocesol{If the sample is large, then the margin of error will be about $1.96 \times 100 / \sqrt{n}$. We want this value to be less than 10, which leads to $n \geq 384.16$, meaning we need a sample size of at least 385 (round up for sample size calculations!).}

% 15
\eocesol{(a)~Two-sided, we are evaluating a difference, not in a particular direction.
(b)~Paired, data are recorded in the same cities at two different time points. The temperature in a city at one point is not independent of the temperature in the same city at another time point.
(c)~$t$-test, sample is small and population standard deviation is unknown.
}

% 17
\eocesol{(a)~Since it's the same students at the beginning and the end of the semester, there is a pairing between the datasets, for a given student their beginning and end of semester grades are dependent.
(b)~Since the subjects were sampled randomly, each observation in the men's group does not have a special correspondence with exactly one observation in the other (women's) group.
(c)~Since it's the same subjects at the beginning and the end of the study, there is a pairing between the datasets, for a subject student their beginning and end of semester artery thickness are dependent.
(d)~Since it's the same subjects at the beginning and the end of the study, there is a pairing between the datasets, for a subject student their beginning and end of semester weights are dependent.}

% 19
\eocesol{(a)~For each observation in one data set, there is exactly one specially-corresponding observation in the other data set for the same geographic location. The data are paired.
(b)~$H_0: \mu_{diff} = 0$ (There is no difference in average daily high temperature between January 1, 1968 and January 1, 2008 in the continental US.) $H_A: \mu_{diff} > 0$ (Average daily high temperature in January 1, 1968 was lower than average daily high temperature in January, 2008 in the continental US.) If you chose a two-sided test, that would also be acceptable. If this is the case, note that your p-value will be a little bigger than what is reported here in part~(d).
(c)~Locations are random and represent less than 10\% of all possible locations in the US. The sample size is at least 30. We are not given the distribution to check the skew. In practice, we would ask to see the data to check this condition, but here we will move forward under the assumption that it is not strongly skewed.
(d)~$T_{50} \approx 1.60 \to 0.05 <$ p-value $< 0.10$.
(e)~Since the p-value $> \alpha$ (since not given use 0.05), fail to reject $H_0$. The data do not provide strong evidence of temperature warming in the continental US. However it should be noted that the p-value is very close to 0.05.
(f)~Type~2 Error, since we may have incorrectly failed to reject $H_0$. There may be an increase, but we were unable to detect it.
(g)~Yes, since we failed to reject $H_0$, which had a null value of 0.}

% 21
\eocesol{(a)~(-0.05, 2.25).
(b)~We are 90\% confident that the average daily high on January 1, 2008 in the continental US was 0.05 degrees lower to 2.25 degrees higher than the average daily high on January 1, 1968.
(c)~No, since 0 is included in the interval.}


\textC{
\end{multicols}
\newpage
\begin{multicols}{2}
}


% 23
\eocesol{(a)~Each of the 36 mothers is related to exactly one of the 36 fathers (and vice-versa), so there is a special correspondence between the mothers and fathers.
(b)~$H_0: \mu_{diff} = 0$. $H_A: \mu_{diff} \ne 0$. Independence: random sample from less than 10\% of population. Sample size of at least 30. The skew of the differences is, at worst, slight. $T_{35} = 2.72 \to$ p-value $= 0.01$. Since p-value $<$ 0.05, reject $H_0$. The data provide strong evidence that the average IQ scores of mothers and fathers of gifted children are different, and the data indicate that mothers' scores are higher than fathers' scores for the parents of gifted children.}

% 25
\eocesol{No, he should not move forward with the test since the distributions of total personal income are very strongly skewed. When sample sizes are large, we can be a bit lenient with skew. However, such strong skew observed in this exercise would require somewhat large sample sizes, somewhat higher than~30.}

% 27
\eocesol{(a)~These data are paired. For example, the Friday the 13th in say, September 1991, would probably be more similar to the Friday the 6th in September 1991 than to Friday the 6th in another month or year.
(b)~Let $\mu_{diff} = \mu_{sixth} - \mu_{thirteenth}$. $H_0: \mu_{diff} = 0$. $H_A: \mu_{diff} \ne 0$.
(c)~Independence: The months selected are not random. However, if we think these dates are roughly equivalent to a simple random sample of all such Friday 6th/13th date pairs, then independence is reasonable. To proceed, we must make this strong assumption, though we should note this assumption in any reported results. Normality: With fewer than 10 observations, we would need to use the $t$-distribution to model the sample mean. The normal probability plot of the differences shows an approximately straight line. There isn't a clear reason why this distribution would be skewed, and since the normal quantile plot looks reasonable, we can mark this condition as reasonably satisfied.
(d)~$T = 4.94$ for $df=10-1=9$ $\to$ p-value $<0.01$.
(e)~Since p-value $<$ 0.05, reject $H_0$. The data provide strong evidence that the average number of cars at the intersection is higher on Friday the 6$^{\text{th}}$ than on Friday the 13$^{\text{th}}$. (We might believe this intersection is representative of all roads, i.e. there is higher traffic on Friday the 6$^{\text{th}}$ relative to Friday the 13$^{\text{th}}$. However, we should be cautious of the required assumption for such a generalization.)
(f)~If the average number of cars passing the intersection actually was the same on Friday the 6$^{\text{th}}$ and $13^{th}$, then the probability that we would observe a test statistic so far from zero is less than 0.01.
(g)~We might have made a Type~1 Error, i.e. incorrectly rejected the null hypothesis.}

% 29
\eocesol{(a)~$H_0: \mu_{diff} = 0$. $H_A: \mu_{diff} \ne 0$. $T=-2.71$. $df=5$. $0.02<$ p-value $<0.05$. Since p-value $<$ 0.05, reject $H_0$. The data provide strong evidence that the average number of traffic accident related emergency room admissions are different between Friday the 6$^{\text{th}}$ and Friday the 13$^{\text{th}}$. Furthermore, the data indicate that the direction of that difference is that accidents are lower on Friday the $6^{th}$ relative to Friday the 13$^{\text{th}}$.
(b)~(-6.49, -0.17).
(c)~This is an observational study, not an experiment, so we cannot so easily infer a causal intervention implied by this statement. It is true that there is a difference. However, for example, this does not mean that a responsible adult going out on Friday the $13^{th}$ has a higher chance of harm than on any other night.}

% 31
\eocesol{(a)~Chicken fed linseed weighed an average of 218.75 grams while those fed horsebean weighed an average of 160.20 grams. Both distributions are relatively symmetric with no apparent outliers. There is more variability in the weights of chicken fed linseed.
(b)~$H_0: \mu_{ls} = \mu_{hb}$. $H_A: \mu_{ls} \ne \mu_{hb}$. We leave the conditions to you to consider. $T=3.02$, $df = min(11, 9) = 9$ $\to$ $0.01<$ p-value $<0.02$. Since p-value $<$ 0.05, reject $H_0$. The data provide strong evidence that there is a significant difference between the average weights of chickens that were fed linseed and horsebean.
(c)~Type~1 Error, since we rejected $H_0$.
(d)~Yes, since p-value $>$ 0.01, we would have failed to reject~$H_0$.}

% 33
\eocesol{$H_0: \mu_C = \mu_S$. $H_A: \mu_C \ne \mu_S$. $T = 3.27$, $df=11$ $\to$ p-value $<0.01$. Since p-value $< 0.05$, reject $H_0$. The data provide strong evidence that the average weight of chickens that were fed casein is different than the average weight of chickens that were fed soybean (with weights from casein being higher). Since this is a randomized experiment, the observed difference can be attributed to the diet.}


\textC{
\end{multicols}
\newpage
\begin{multicols}{2}
}


% 35
\eocesol{$H_0: \mu_{T} = \mu_{C}$. $H_A: \mu_{T} \ne \mu_{C}$. $T=2.24$, $df=21$ $\to$ $0.02<$ p-value $<0.05$. Since p-value $<$ 0.05, reject $H_0$. The data provide strong evidence that the average food consumption by the patients in the treatment and control groups are different. Furthermore, the data indicate patients in the distracted eating (treatment) group consume more food than patients in the control group.}

% 37
\eocesol{Let $\mu_{diff} = \mu_{pre} - \mu_{post}$. $H_0: \mu_{diff} = 0$: Treatment has no effect. $H_A: \mu_{diff} > 0$: Treatment is effective in reducing P.D.T. scores, the average pre-treatment score is higher than the average post-treatment score. Note that the reported values are pre minus post, so we are looking for a positive difference, which would correspond to a reduction in the P.D.T. score. Conditions are checked as follows. Independence: The subjects are randomly assigned to treatments, so the patients in each group are independent. All three sample sizes are smaller than 30, so we use $t$-tests. Distributions of differences are somewhat skewed. The sample sizes are small, so we cannot reliably relax this assumption. (We will proceed, but we would not report the results of this specific analysis, at least for treatment group 1.) For all three groups: $df=13$. $T_1= 1.89$ ($0.025<$ p-value $<0.05$), $T_2=1.35$ (p-value = 0.10), $T_3 = -1.40$ (p-value $>0.10$). The only significant test reduction is found in Treatment 1, however, we had earlier noted that this result might not be reliable due to the skew in the distribution. Note that the calculation of the p-value for Treatment 3 was unnecessary: the sample mean indicated a increase in P.D.T. scores under this treatment (as opposed to a decrease, which was the result of interest). That is, we could tell without formally completing the hypothesis test that the p-value would be large for this treatment group.}

% 39
\eocesol{Difference we care about: 40. Single tail of 90\%: $1.28 \times SE$. Rejection region bounds: $\pm 1.96 \times SE$ (if 5\% significance level). Setting $3.24 \times SE = 40$, subbing in $SE = \sqrt{\frac{94^2}{n} + \frac{94^2}{n}}$, and solving for the sample size $n$ gives 116 plots of land for each fertilizer.}

% 41

\eocesol{Alternative.}

% 43

\eocesol{$H_0$: $\mu_1 = \mu_2 = \cdots = \mu_6$. $H_A$: The average weight varies across some (or all) groups. Independence: Chicks are randomly assigned to feed types (presumably kept separate from one another), therefore independence of observations is reasonable. Approx. normal: the distributions of weights within each feed type appear to be fairly symmetric. Constant variance: Based on the side-by-side box plots, the constant variance assumption appears to be reasonable. There are differences in the actual computed standard deviations, but these might be due to chance as these are quite small samples. $F_{5,65} = 15.36$ and the p-value is approximately 0. With such a small p-value, we reject $H_0$. The data provide convincing evidence that the average weight of chicks varies across some (or all) feed supplement groups.}

% 45
\eocesol{(a)~$H_0$: The population mean of MET for each group is equal to the others. $H_A$: At least one pair of means is different.
(b)~Independence: We don't have any information on how the data were collected, so we cannot assess independence. To proceed, we must assume the subjects in each group are independent. In practice, we would inquire for more details. Approx. normal: The data are bound below by zero and the standard deviations are larger than the means, indicating very strong skew. However, since the sample sizes are extremely large, even extreme skew is acceptable. Constant variance: This condition is sufficiently met, as the standard deviations are reasonably consistent across groups.
(c)~See below, with the last column omitted:\\[-2mm]
\begin{adjustwidth}{-4em}{-4em}
{\tiny
\begin{center}
\renewcommand{\arraystretch}{1.25}
\begin{tabular}{lrrrr}
  \hline
            & Df    & Sum Sq        & Mean Sq   & F value \\
  \hline
coffee      & {\textcolor{oiB}{{\scriptsize 4}}}     & {\textcolor{oiB}{{\scriptsize 10508}}}       & {\textcolor{oiB}{{\scriptsize 2627}}}             & {\textcolor{oiB}{{\scriptsize 5.2}}} \\
Residuals       & {\textcolor{oiB}{{\scriptsize 50734}}} & 25564819     & {\textcolor{oiB}{{\scriptsize  504}}}         &  \\
   \hline
Total           & {\textcolor{oiB}{{\scriptsize 50738}}} & 25575327 \\
\hline
\end{tabular}
\end{center}
}
\end{adjustwidth} \vspace{1mm}
(d)~Since p-value is very small, reject $H_0$. The data provide convincing evidence that the average MET differs between at least one pair of groups.}


\textC{
\end{multicols}
\newpage
\begin{multicols}{2}
}


% 47
\eocesol{(a)~$H_0$: Average GPA is the same for all majors. $H_A$: At least one pair of means are different.
(b)~Since p-value $>$ 0.05, fail to reject $H_0$. The data do not provide convincing evidence of a difference between the average GPAs across three groups of majors.
(c)~The total degrees of freedom is $195 + 2 = 197$, so the sample size is $197+1=198$.}

% 49
\eocesol{(a)~False. As the number of groups increases, so does the number of comparisons and hence the modified significance level decreases.
(b)~True.
(c)~True.
(d)~False. We need observations to be independent regardless of sample size.}

% 51
\eocesol{(a)~$H_0$: Average score difference is the same for all treatments. $H_A$: At least one pair of means are different.
(b)~We should check conditions. If we look back to the earlier exercise, we will see that the patients were randomized, so independence is satisfied. There are some minor concerns about skew, especially with the third group, though this may be acceptable. The standard deviations across the groups are reasonably similar. Since the p-value is less than 0.05, reject $H_0$. The data provide convincing evidence of a difference between the average reduction in score among treatments.
(c)~We determined that at least two means are different in part (b), so we now conduct $K=3\times2/2=3$ pairwise $t$-tests that each use $\alpha=0.05/3 = 0.0167$ for a significance level. Use the following hypotheses for each pairwise test. $H_0$: The two means are equal. $H_A$: The two means are different. The sample sizes are equal and we use the pooled SD, so we can compute $SE=3.7$ with the pooled $df=39$. The p-value only for Trmt 1 vs. Trmt 3 may be statistically significant: $0.01<$ p-value $<0.02$. Since we cannot tell, we should use a computer to get the p-value, 0.015, which is statistically significant for the adjusted significance level. That is, we have identified Treatment 1 and Treatment 3 as having different effects. Checking the other two comparisons, the differences are not statistically significant.}

\end{multicols}
