% -*- compile-command: "cd ../ && make" -*-
\eocesolch{Inference for categorical data}

%%%%%%%%%%%%%%%%%%%%%%%%%%%%%

\begin{multicols}{2}

% 1
\eocesol{(a)~False. Doesn't satisfy success-failure condition.
(b)~True. The success-failure condition is not satisfied. In most samples we would expect $\hat{p}$ to be close to 0.08, the true population proportion. While $\hat{p}$ can be much above 0.08, it is bound below by 0, suggesting it would take on a right skewed shape. Plotting the sampling distribution would confirm this suspicion.
(c)~False. $SE_{\hat{p}} = 0.0243$, and $\hat{p} = 0.12$ is only $\frac{0.12 - 0.08}{0.0243} = 1.65$ SEs away from the mean, which would not be considered unusual.
(d)~True. $\hat{p}=0.12$ is 2.32 standard errors away from the mean, which is often considered unusual.
(e)~False. Decreases the SE by a factor of $1/\sqrt{2}$.}

% 3
\eocesol{(a)~True. See the reasoning of 6.1(b).
(b)~True. We take the square root of the sample size in the SE formula.
(c)~True. The independence and success-failure conditions are satisfied.
(d)~True. The independence and success-failure conditions are satisfied.}

% 5
\eocesol{(a)~False. A confidence interval is constructed to estimate the population proportion, not the sample proportion.
(b)~True. 95\% CI: $70\%\ \pm\ 8\%$.
(c)~True. By the definition of the confidence level.
(d)~True. Quadrupling the sample size decreases the SE and ME by a factor of $1/\sqrt{4}$.
(e)~True. The 95\% CI is entirely above 50\%.}

% 7
\eocesol{With a random sample from $<10\%$ of the population, independence is satisfied. The success-failure condition is also satisfied. $ME = z^{\star} \sqrt{ \frac{\hat{p} (1-\hat{p})} {n} } = 1.96 \sqrt{ \frac{0.56 \times  0.44}{600} }= 0.0397 \approx 4\%$}


\textC{
\end{multicols}
\newpage
\begin{multicols}{2}
}


% 9
\eocesol{(a)~Proportion of graduates from this university who found a job within one year of graduating. $\hat{p} = 348/400 = 0.87$.
(b)~This is a random sample from less than 10\% of the population, so the observations are independent. Success-failure condition is satisfied: 348 successes, 52 failures, both well above~10.
(c)~(0.8371, 0.9029). We are 95\% confident that approximately 84\% to 90\% of graduates from this university found a job within one year of completing their undergraduate degree.
(d)~95\% of such random samples would produce a 95\% confidence interval that includes the true proportion of students at this university who found a job within one year of graduating from college.
(e)~(0.8267, 0.9133). Similar interpretation as before.
(f)~99\% CI is wider, as we are more confident that the true proportion is within the interval and so need to cover a wider range.}

% 11
\eocesol{(a)~No. The sample only represents students who took the SAT, and this was also an online survey.
(b)~(0.5289, 0.5711). We are 90\% confident that 53\% to 57\% of high school seniors who took the SAT are fairly certain that they will participate in a study abroad program in college.
(c)~90\% of such random samples would produce a 90\% confidence interval that includes the true proportion.
(d)~Yes. The interval lies entirely above 50\%.}

% 13
\eocesol{(a)~This is an appropriate setting for a hypothesis test. $H_0: p = 0.50$. $H_A:  p > 0.50$. Both independence and the success-failure condition are satisfied. $Z=1.12$ $\to$ p-value $= 0.1314$. Since the p-value $> \alpha=0.05$, we fail to reject $H_0$. The data do not provide strong evidence that more than half of all Independents oppose the public option plan.
(b)~Yes, since we did not reject $H_0$ in part~(a).}

% 15
\eocesol{(a)~$H_0: p = 0.38$. $H_A: p \ne 0.38$. Independence (random sample, $<10\%$ of population) and the success-failure condition are satisfied. $Z=-20.5$ $\to$ p-value $\approx 0$. Since the p-value is very small, we reject $H_0$. The data provide strong evidence that the proportion of Americans who only use their cell phones to access the internet is different than the Chinese proportion of 38\%, and the data indicate that the proportion is lower in the US.
(b)~If in fact 38\% of Americans used their cell phones as a primary access point to the internet, the probability of obtaining a random sample of 2,254 Americans where 17\% or less or 59\% or more use their only their cell phones to access the internet would be approximately 0.
(c)~(0.1545, 0.1855). We are 95\% confident that approximately 15.5\% to 18.6\% of all Americans primarily use their cell phones to browse the internet.}

% 17
\eocesol{(a)~$H_0: p = 0.5$. $H_A: p > 0.5$. Independence (random sample, $<10\%$ of population) is satisfied, as is the success-failure conditions (using $p_0 = 0.5$, we expect 40 successes and 40 failures). $Z = 2.91$ $\to$ p-value $= 0.0018$. Since the p-value $< 0.05$, we reject the null hypothesis. The data provide strong evidence that the rate of correctly identifying a soda for these people is significantly better than just by random guessing.
(b)~If in fact people cannot tell the difference between diet and regular soda and they randomly guess, the probability of getting a random sample of 80 people where 53 or more identify a soda correctly would be 0.0018.}

% 19
\eocesol{(a)~Independence is satisfied (random sample from $<10\%$ of the population), as is the success-failure condition (40 smokers, 160 non-smokers). The 95\% CI: (0.145, 0.255). We are 95\% confident that 14.5\% to 25.5\% of all students at this university smoke.
(b)~We want $z^{\star}SE$ to be no larger than 0.02 for a 95\% confidence level. We use $z^{\star}=1.96$ and plug in the point estimate $\hat{p}=0.2$ within the SE formula: $1.96\sqrt{0.2(1-0.2)/n} \leq 0.02$. The sample size $n$ should be at least 1,537.}

% 21
\eocesol{The margin of error, which is computed as $z^{\star}SE$, must be smaller than 0.01 for a 90\% confidence level. We use $z^{\star} = 1.65$ for a 90\% confidence level, and we can use the point estimate $\hat{p}=0.52$ in the formula for $SE$. $1.65\sqrt{0.52(1-0.52)/n} \leq 0.01$. Therefore, the sample size $n$ must be at least 6,796.}


\textC{
\end{multicols}
\newpage
\begin{multicols}{2}
}


% 23
\eocesol{This is not a randomized experiment, and it is unclear whether people would be affected by the behavior of their peers. That is, independence may not hold. Additionally, there are only 5 interventions under the provocative scenario, so the success-failure condition does not hold. Even if we consider a hypothesis test where we pool the proportions, the success-failure condition will not be satisfied. Since one condition is questionable and the other is not satisfied, the difference in sample proportions will not follow a nearly normal distribution.}

% 25
\eocesol{(a)~False. The entire confidence interval is above 0.
(b)~True.
(c)~True.
(d)~True.
(e)~False. It is simply the negated and reordered values: \mbox{(-0.06,-0.02)}.}

% 27
\eocesol{(a)~(0.23, 0.33). We are 95\% confident that the proportion of Democrats who support the plan is 23\% to 33\% higher than the proportion of Independents who do.
(b)~True.}

% 29
\eocesol{(a)~College grads: 23.7\%. Non-college grads: 33.7\%.
(b)~Let $p_{CG}$ and $p_{NCG}$ represent the proportion of college graduates and non-college graduates who responded ``do not know". $H_0: p_{CG} = p_{NCG}$. $H_A: p_{CG} \ne p_{NCG}$. Independence is satisfied (random sample, $<10\%$ of the population), and the success-failure condition, which we would check using the pooled proportion ($\hat{p} = 235/827 = 0.284$), is also satisfied. $Z=-3.18$ $\to$ p-value = 0.0014. Since the p-value is very small, we reject $H_0$. The data provide strong evidence that the proportion of college graduates who do not have an opinion on this issue is different than that of non-college graduates. The data also indicate that fewer college grads say they ``do not know'' than non-college grads (i.e. the data indicate the direction after we reject $H_0$).}

% 31
\eocesol{(a)~College grads: 35.2\%. Non-college grads: 33.9\%.
(b)~Let $p_{CG}$ and $p_{NCG}$ represent the proportion of college graduates and non-college grads who support offshore drilling. $H_0: p_{CG} = p_{NCG}$. $H_A: p_{CG} \ne p_{NCG}$. Independence is satisfied (random sample, $<10\%$ of the population), and the success-failure condition, which we would check using the pooled proportion ($\hat{p} = 286/827 = 0.346$), is also satisfied. $Z = 0.39$ $\to$ p-value $=0.6966$. Since the p-value $> \alpha$ (0.05), we fail to reject $H_0$. The data do not provide strong evidence of a difference between the proportions of college graduates and non-college graduates who support off-shore drilling in California.}

% 33
\eocesol{Subscript $_C$ means control group. Subscript $_T$ means truck drivers. $H_0: p_C = p_T$. $H_A: p _C \ne p_T$. Independence is satisfied (random samples, $<10\%$ of the population), as is the success-failure condition, which we would check using the pooled proportion ($\hat{p} = 70/495 = 0.141$). $Z = -1.65$ $\to$ p-value $=0.0989$. Since the p-value is high (default to $\alpha = 0.05$), we fail to reject $H_0$. The data do not provide strong evidence that the rates of sleep deprivation are different for non-transportation workers and truck drivers.}

% 35
\eocesol{(a)~Summary of the study:
\begin{center}\scriptsize
\begin{tabular}{l l c c c}
                                &           & \multicolumn{2}{c}{\textit{Virol. failure}}   &       \\
\cline{3-4}
                                &           & Yes       & No        & Total \\
\cline{2-5}
\multirow{2}{*}{\textit{Treatment}}     & Nevaripine    & 26            & 94        & 120   \\
                                & Lopinavir & 10            & 110   & 120   \\
\cline{2-5}
                                & Total     & 36            & 204   & 240
\end{tabular}
\end{center}
(b)~$H_0: p_N = p_L$. There is no difference in virologic failure rates between the Nevaripine and Lopinavir groups. $H_A: p_N \ne p_L$. There is some difference in virologic failure rates between the Nevaripine and Lopinavir groups.
(c)~Random assignment was used, so the observations in each group are independent. If the patients in the study are representative of those in the general population (something impossible to check with the given information), then we can also confidently generalize the findings to the population. The success-failure condition, which we would check using the pooled proportion ($\hat{p} = 36/240 = 0.15$), is satisfied. $Z=2.89$ $\to$ p-value $=0.0039$. Since the p-value is low, we reject $H_0$. There is strong evidence of a difference in virologic failure rates between the Nevaripine and Lopinavir groups do not appear to be independent.}

% 37
\eocesol{No. The samples at the beginning and at the end of the semester are not independent since the survey is conducted on the same students.}

% 39
\eocesol{(a)~False. The chi-square distribution has one parameter called degrees of freedom.
(b)~True.
(c)~True.
(d)~False. As the degrees of freedom increases, the shape of the chi-square distribution becomes more symmetric.}


\textC{
\end{multicols}
\newpage
\begin{multicols}{2}
}


% 41
\eocesol{(a)~$H_0$: The distribution of the format of the book used by the students follows the professor's predictions. $H_A$: The distribution of the format of the book used by the students does not follow the professor's predictions.
(b)~$E_{hard~copy} = 126 \times  0.60 = 75.6$. $E_{print} = 126 \times  0.25 = 31.5$. $E_{online} = 126 \times  0.15 = 18.9$.
(c)~Independence:  The sample is not random. However, if the professor has reason to believe that the proportions are stable from one term to the next and students are not affecting each other's study habits, independence is probably reasonable. Sample size: All expected counts are at least 5.
(d)~$\chi^2 = 2.32$, $df=2$, p-value $> 0.3$.
(e)~Since the p-value is large, we fail to reject $H_0$. The data do not provide strong evidence indicating the professor's predictions were statistically inaccurate.}

% 43
\eocesol{Use a chi-squared goodness of fit test.
$H_0$: Each option is equally likely.
$H_A$: Some options are preferred over others.
Total sample size: 99.
Expected counts: (1/3) * 99 = 33 for each option. These are all above 5, so conditions are satisfied.
$df = 3 - 1 = 2$ and $\chi^2 = \frac{(43 - 33)^2}{33} + \frac{(21 - 33)^2}{33} + \frac{(35 - 33)^2}{33} = 7.52 \rightarrow 0.02 <$ p-value $< 0.05$. Since the p-value is less than 5\%, we reject $H_0$. The data provide convincing evidence that some options are preferred over others.}

% 45
\eocesol{(a) Two-way table:
\begin{center}\scriptsize
\begin{tabular}{l l c c c}
& \multicolumn{2}{c}{\textit{Quit}} &       \\
\cline{2-3}
\textit{Treatment}      & Yes       & No        & Total \\
\hline
Patch + support group   & 40            & 110   & 150   \\
Only patch          & 30            & 120   & 150   \\
\cline{1-4}
Total               & 70            & 230   & 300 \\
\cline{1-4}
\end{tabular}
\end{center}
(b-i)~$E_{row_1, col_1} = \frac{(row~1~total)\times(col~1~total)}{table~total} = 35$. This is lower than the observed value. \\
(b-ii)~$E_{row_2, col_2} = \frac{(row~2~total)\times(col~2~total)}{table~total} = 115$. This is lower than the observed value.}

% 47
\eocesol{$H_0$: The opinion of college grads and non-grads is not different on the topic of drilling for oil and natural gas off the coast of California. $H_A$: Opinions regarding the drilling for oil and natural gas off the coast of California has an association with earning a college degree.
\begin{align*}
&E_{row~1, col~1} = 151.5 && E_{row~1, col~2} = 134.5 \\
&E_{row~2, col~1} = 162.1 && E_{row~2, col~2} = 143.9 \\
&E_{row~3, col~1} = 124.5 && E_{row~3, col~2} = 110.5
\end{align*}
Independence: The samples are both random, unrelated, and from less than 10\% of the population, so independence between observations is reasonable. Sample size: All expected counts are at least 5.
$\chi^2 = 11.47$, $df = 2$ $\to$ $0.001<$ p-value $<0.005$.
Since the p-value $< \alpha$, we reject $H_0$.  There is strong evidence that there is an association between support for off-shore drilling and having a college degree.}

% 49
\eocesol{(a)~$H_0$:~The age of Los Angeles residents is independent of shipping carrier preference variable. $H_A$:~The age of Los Angeles residents is associated with the shipping carrier preference variable. (b)~The conditions are not satisfied since some expected counts are below~5.}

% 51
\eocesol{No. For a confidence interval, we check the success-failure condition using the data, and there are only 9 respondents who said bullying is no problem at all.}

% 53
\eocesol{(a)~$H_0: p = 0.69$. $H_A: p \ne 0.69$.
(b)~$\hat{p} = \frac{17}{30} = 0.57$.
(c)~The success-failure condition is not satisfied; note that it is appropriate to use the null value ($p_0 = 0.69$) to compute the expected number of successes and failures.
(d)~Answers may vary. Each student can be represented with a card. Take 100 cards, 69 black cards representing those who follow the news about Egypt and 31 red cards representing those who do not. Shuffle the cards and draw with replacement (shuffling each time in between draws) 30 cards representing the 30 high school students. Calculate the proportion of black cards in this sample, $\hat{p}_{sim}$, i.e. the proportion of those who follow the news in the simulation. Repeat this many times (e.g. 10,000 times) and plot the resulting sample proportions. The p-value will be two times the proportion of simulations where $\hat{p}_{sim} \le 0.57$. (Note: we would generally use a computer to perform these simulations.)
(e)~The p-value is about $0.001 + 0.005 + 0.020 + 0.035 + 0.075 = 0.136$, meaning the two-sided p-value is about 0.272. Your p-value may vary slightly since it is based on a visual estimate. Since the p-value is greater than 0.05, we fail to reject $H_0$. The data do not provide strong evidence that the proportion of high school students who followed the news about Egypt is different than the proportion of American adults who did.}


\textC{
\end{multicols}
\newpage
\begin{multicols}{2}
}


% 55
\eocesol{The subscript $_{pr}$ corresponds to provocative and $_{con}$ to conservative.
(a)~$H_0: p_{pr} = p_{con}$. $H_A: p_{pr} \ne p_{con}$.
(b)~-0.35.
(c)~The left tail for the p-value is calculated by adding up the two left bins: $0.005+0.015=0.02$. Doubling the one tail, the p-value is 0.04. (Students may have approximate results, and a small number of students may have a p-value of about 0.05.) Since the p-value is low, we reject $H_0$. The data provide strong evidence that people react differently under the two scenarios.}

\end{multicols}
